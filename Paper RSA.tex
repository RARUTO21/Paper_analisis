\documentclass[conference,compsoc, 10pt]{IEEEtran}

% *** CITATION PACKAGES ***
%
\ifCLASSOPTIONcompsoc
  \usepackage[nocompress]{cite}
\else
  \usepackage{cite}
\fi

\ifCLASSINFOpdf
\else
\fi

\hyphenation{op-tical net-works semi-conduc-tor}


\begin{document}	

\title{RSA Algorithm\\ Algorithm Analysis}
\author{\IEEEauthorblockN{Geovanny Burgos Retana}
\IEEEauthorblockA{Computer Engineering Student\\
Instituto Tecnologico de Costa Rica\\
San Jose, Costa Rica\\
Email: geoburgosretana@gmail.com}
\and
\IEEEauthorblockN{Anthony Leandro}
\IEEEauthorblockA{Computer Engineering Student\\
Instituto Tecnologico de Costa Rica\\
Limon, Costa Rica\\
Email: Anthonylle@hotmail.com}
\and
\IEEEauthorblockN{Bryan Mena Villalobos}
\IEEEauthorblockA{Computer Engineering Student\\
Instituto Tecnologico de Costa Rica\\
Heredia, Costa Rica\\
Email: mena97villalobos@gmail.com}}

\maketitle
\large
\begin{abstract}
	\large
	The abstract goes here.
\end{abstract}

\IEEEpeerreviewmaketitle

\section{Introduction}
RSA algorithm, first published on 1977 by Ron Rivest, Adi Shamir, and Len Adleman,  is a public-key cryptosystem, in this days, often used in various web servers, browsers and commercial system to protect web traffic among some other things like email encryption In this type of cryptosystem the encryption key and the decryption key are different also, the encryption is public while de decryption key is private. This cryptosystem is base on the difficulty of factoring to prime large numbers.

\section{The Algorithm}
Taking the words of Weisstein, E RSA encryption defines as "A public-key algorithm which uses prime factorization as the trapdoor one-way function", 

\subsection{Before RSA}
RSA, in a certain way, was the first implementation of public key encryption but as seen in \textbf{\textit{The history of Non-Secret Encryption}} by J.H. Ellis public key encryption was long before been developed, in 1970 J. H. Ellis conceive a non-secret digital encryption(today known as public key encryption), in this time Ellis couldn't see a way to implement this type of encryption but in 1973 an employee of GCHQ came with the basic idea of RSA encryption base on Ellis work, but, as the new techniques discover on GCHQ that are potentially harmful, by definition, are classified information, this implementation was kept on secret.

\subsection{Actual panorama}
With the breakthrough in quantum computers is known that Shor's algorithm broke a 768-bit key (usually key sizes variate from 1024 to 4096 bits), this give us an idea of how quantum computers may change our way of thinking

\section{Methodology}
Bla bla bla

\section{Experiments}
as

\section{Analysis and Results}
The analysis goes here

\section{Conclusion}
The conclusion goes here.

\large
\begin{thebibliography}{1}
\bibitem{IEEEhowto:kopka}
Boneh, D (November, 1998). Twenty Years of Attacks on the RSA Cryptosystem, Retrieve from: http://crypto.stanford.edu/~dabo/pubs/papers/RSA-survey.pdf

\bibitem{IEEEhowto:kopta}
Ellis, J.H. (January, 1970). The possibility of secure non-secret analogue encryption, Retrieve from: http://cryptocellar.org/cesg/possnse.pdf

\bibitem{IEEEhowto:kopta}
Ellis, J.H. (May, 1970). The possibility of secure non-secret analogue encryption, Retrieve from: https://www.gchq.gov.uk/file/cesgresearchreportno3007pdf-2

\bibitem{IEEEhowto:koopta}
Ellis, J.H. 1987. The History of Non-Secret Encryption. Retrieve from: https://web.archive.org/web/20130404174201/\newline
https://cryptocellar.web.cern.ch/cryptocellar/cesg/ellis.pdf

\bibitem{IEEEhowto:kopka}
Weisstein, Eric W. "RSA Encryption." From MathWorld--A Wolfram Web Resource. Retrieve from: http://mathworld.wolfram.com/RSAEncryption.html

\bibitem{IEEEhowto:kopka}
R.L.Rivest, A.Sharmir, L.Adleman: A method for obtaining digital signatures and public key Cryptosystems”, Tata McGraw-Hill Retrieve from: http://people.csail.mit.edu/rivest/Rsapaper.pdf

\bibitem{IEEEhowto:kopta}
Williamson, Malcolm J. (January 21, 1974). Non—secret encryption using a finite field. Retrieve from:
https://www.gchq.gov.uk/sites/default/files/document\_files/\newline nonsecret\_encryption\_finite\_field\_0.pdf

\end{thebibliography}

\end{document}


